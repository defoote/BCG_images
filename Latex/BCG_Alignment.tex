\documentclass[a4paper,12pt]{article}
\usepackage{indentfirst}
\usepackage{cite}

\begin{document}

\title{Comparing BCG and Merger Axes}
\author{Drake Foote}
\date{\today}
\maketitle
\section{Introduction}
	Galaxy clusters are some of the largest structures in the universe. With potentially thousands of galaxies in one cluster it seems unlikely that any single galaxy may be significant or characteristic to the cluster as a whole. However, it has been shown that the brightest cluster galaxy (BCG), which is the most luminous galaxy in the cluster, can tell us a lot about the structure and evolution of the cluster. Evidence has been found for a few relationships between BCGs and the clusters in which they are contained. First, BCGs tend to be in the center of their parent cluster. Also, it appears BCGs are formed through the hierarchical merging of smaller galaxies in the cluster, and are thus elliptical. Therefore, BCGs are often one of the most massive galaxies in the cluster. Lastly, and most importantly for this paper is the relationship between the BCGs major axis and the major axis of the cluster. It has been well proven that the galaxies within a cluster tend to align with the cluster's major axis. More so, the BCGs within clusters have a stronger alignment with the cluster \cite{West}, emphasizing the importance of the BCG in the cluster. This raises the question of how BCGs came to be aligned with their clusters. 
\par 
	One possible explanation for this alignment is cluster mergers. Cluster mergers are the most energetic interactions in the universe today. Clusters will undergo multiple mergers in their lifetime, each one having a great effect on their structure. This hierarchical model of cluster formation tells us that mergers have prfound effects on the clusters in them. Mergers in a clusters history can have great effects on how that cluster is shaped, causing the cluster to become elongated along the merger axis and the cluster to align along the merger axis. Mergers also drastically affect the galaxies within each cluster. For this reason it is possible that they may have some effect on how the galaxies within them align to their surroundings. If the alignment of BCGs with their parent cluster were caused by mergers, one would expect BCGs in merging clusters to align with the merger axis. To analyze this we looked at BCGs within post-pericenter mergers to look for preferential alignment along the merger axis. From a sample of 22 merging systems, with 44 merging cluster in total, we aimed to examine this relationship between the alignment of the BCG and merger axis. 	
\section{Methods}

\section{Results}
\section{Discussion}

\bibliography{references}
\bibliographystyle{plain}

\end{document}
